% ⚠️ LEGACY: Этот файл мигрирован в 02_networking/08_tcp_client_server/netcat/network_file_transfer_theory.tex
% Этот файл сохранен для обратной совместимости и будет удален после завершения миграции.
% Используйте новый путь: 02_networking/08_tcp_client_server/netcat/network_file_transfer_theory.tex

\documentclass[12pt]{article}
\usepackage[utf8]{inputenc}
\usepackage[russian]{babel}
\usepackage{amsmath}
\usepackage{listings}
\usepackage{geometry}
\usepackage{graphicx}

\geometry{a4paper, margin=1in}

\lstset{
    language=C++,
    basicstyle=\ttfamily\small,
    keywordstyle=\color{blue},
    stringstyle=\color{red},
    commentstyle=\color{green!50!black},
    numbers=left,
    numberstyle=\tiny,
    stepnumber=1,
    numbersep=5pt,
    backgroundcolor=\color{white},
    showspaces=false,
    showstringspaces=false,
    showtabs=false,
    frame=single,
    tabsize=2,
    captionpos=b,
    breaklines=true,
    breakatwhitespace=false,
    title=\lstname
}

\title{Сетевая передача файлов в C++: Теория и практика}
\author{Кафедра системного программирования}
\date{}

\begin{document}

\maketitle

\section{Введение в сетевое программирование}

\subsection{Базовые концепции сокетов}

Сокет представляет собой программный интерфейс для обмена данными между процессами, работающими на разных устройствах в сети. В контексте C++ сокеты являются абстракцией над низкоуровневыми сетевыми протоколами.

\subsubsection{Типы сокетов}
\begin{itemize}
    \item \textbf{TCP (SOCK\_STREAM)}: Надежное, упорядоченное соединение
    \item \textbf{UDP (SOCK\_DGRAM)}: Быстрая, но ненадежная передача данных
\end{itemize}

\subsection{Математическая модель передачи}

Время передачи файла может быть описано следующей формулой:

\begin{equation}
T_{transfer} = \frac{S_{file}}{B_{network}}
\end{equation}

где:
\begin{itemize}
    \item $T_{transfer}$ - время передачи
    \item $S_{file}$ - размер файла (байты)
    \item $B_{network}$ - пропускная способность сети (байты/сек)
\end{itemize}

\section{Механизмы буферизации}

\begin{lstlisting}[caption={Механизм буферной передачи файла}]
const size_t BUFFER_SIZE = 4096;  // Оптимальный размер буфера

void transfer_file_with_buffer(std::ifstream& input_file, int socket) {
    char buffer[BUFFER_SIZE];
    
    while (!input_file.eof()) {
        input_file.read(buffer, BUFFER_SIZE);
        std::streamsize bytes_read = input_file.gcount();
        
        if (bytes_read > 0) {
            send(socket, buffer, bytes_read, 0);
        }
    }
}
\end{lstlisting}

\section{Сравнение размеров буфера}

\begin{table}[h]
\centering
\begin{tabular}{|c|c|c|}
\hline
Размер буфера & Преимущества & Недостатки \\
\hline
1 КБ & Низкое потребление памяти & Частые системные вызовы \\
4 КБ & Баланс производительности & Умеренное использование памяти \\
64 КБ & Высокая пропускная способность & Больше памяти, фрагментация \\
\hline
\end{tabular}
\caption{Характеристики размеров буфера}
\end{table}

\section{Рекомендации по оптимизации}

\begin{enumerate}
    \item Использовать размер буфера, кратный размеру страницы памяти
    \item Учитывать особенности сетевого стека ОС
    \item Тестировать производительность на реальных данных
\end{enumerate}

\section{Заключение}

Сетевая передача файлов - сложный процесс, требующий глубокого понимания:
\begin{itemize}
    \item Низкоуровневых механизмов операционной системы
    \item Сетевых протоколов
    \item Эффективного управления памятью
\end{itemize}

\section*{Рекомендуемая литература}
\begin{itemize}
    \item ``Unix Network Programming'' - W. Richard Stevens
    \item ``TCP/IP Illustrated'' - W. Richard Stevens
    \item ``Modern C++ Design'' - Andrei Alexandrescu
\end{itemize}

\end{document} 